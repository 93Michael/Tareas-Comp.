\documentclass{article}
\usepackage[utf8]{inputenc}
\usepackage[spanish]{babel}
\usepackage{graphicx}

\title{Informe de Movimiento Rectilíneo Uniforme}
\author{Chambi Quispe Aded M.}
\date{\25/04/2024}

\begin{document}

\maketitle

\section{Introducción}
En este informe se estudiará el Movimiento Rectilíneo Uniforme (MRU), que es aquel en el cual un objeto se mueve en línea recta con velocidad constante.

\section{Metodología}
Para estudiar el MRU, se utilizó un carrito sobre una superficie horizontal con baja fricción. Se marcaron distancias regulares en la pista y se registraron los tiempos de paso del carrito.

\section{Resultados}
Se obtuvo una tabla de tiempos de paso y distancias recorridas por el carrito, que se muestra a continuación:

\begin{table}[h]
\centering
\begin{tabular}{|c|c|}
\hline
Tiempo (s) & Distancia (m) \\ \hline
0          & 0             \\ \hline
1          & 2             \\ \hline
2          & 4             \\ \hline
3          & 6             \\ \hline
4          & 8             \\ \hline
\end{tabular}
\caption{Datos experimentales del MRU}
\end{table}

\section{Análisis de Resultados}
Los datos experimentales muestran que el carrito recorre distancias iguales en intervalos de tiempo iguales, confirmando el comportamiento de MRU.

\section{Ejemplo}
Supongamos que un automóvil se desplaza a una velocidad constante de $20$ metros por segundo durante $5$ segundos. Si su posición inicial es $x_0 = 0$ metros, podemos calcular su posición final utilizando la fórmula del M.R.U.:
\[ x = 0 + (20 \times 5) = 100 \text{ metros} \]

\section{Conclusiones}
Se concluye que el carrito se movió con movimiento rectilíneo uniforme, ya que mantuvo una velocidad constante a lo largo del tiempo.

\section{Imágenes}
A continuación se muestra una imagen del experimento:

\begin{figure}[h]
\centering
\includegraphics[width=0.5\textwidth]{A1.jpg}
\caption{Experimento de MRU}
\label{fig:carrito_mru}
\end{figure}

\begin{table}[h]
\centering
\begin{tabular}{|c|c|}
\hline
\multicolumn{2}{|c|}{\textbf{SIMBOLISMOS}} \\ \hline
\textbf{VARIABLES} & \textbf{SIGNIFICADO} \\ \hline
$X$ & La posición final del objeto \\ \hline
$X_0$ & La posición inicial del objeto     \\ \hline
$v$ & La velocidad del objeto   \\ \hline
$t$ & El tiempo que pasará \\ \hline

\end{tabular}
\caption{DESCRIPCIÓNDE LAS VARUABLES}
\label{tab:horario}
\end{table}

\section{bibliografia}
 https://www.fisicalab.com/amp/apartado/mru

 https://edu.gcfglobal.org/es/movimiento/movimiento-rectilineo-uniforme/1/
 
\end{document}
